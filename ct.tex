%% LyX 2.1.4 created this file.  For more info, see http://www.lyx.org/.
%% Do not edit unless you really know what you are doing.
\documentclass[11pt]{article}
\usepackage[T1]{fontenc}
\usepackage[utf8]{inputenc}
\usepackage{wasysym}

\makeatletter
%%%%%%%%%%%%%%%%%%%%%%%%%%%%%% User specified LaTeX commands.
%  CATEGORY THEORY 2016
%
%  LaTeX format for the abstracts
%
%  January, 2016



\hoffset=-10.5mm
\addtolength{\textwidth}{16mm}
\voffset=-20mm
\addtolength{\textheight}{35mm}

\renewcommand{\baselinestretch}{1.2}

\def\thefootnote{\fnsymbol{footnote}}

\makeatother

\begin{document}
\setcounter{page}{1}

\begin{center}
\textsf{\LARGE{}Polymorphism} \textsf{\setcounter{footnote}{0}}
\par\end{center}

\textit{Short}. 1. This \texttt{https://github.com/1337777/borceux/}
solves some question of Ahrens and Kan-Riehl, which is how to program
Kelly's «enriched categories» and how the inter-dependence of «naturality»
with «category» is cyclic. Also this attempts to clarify the contrast
«categorical algebra» (ring\slash locale-presentation and its ``internal
logic''), from «categorial logic» in the style of the «enriched\slash
encoded\slash programmed\slash recursion» categories of Kelly-Dosen
or Lawvere-Lambek, for example : the yoneda lemma and most categorial
lemmas are no-more-than Gentzen's constructive logic of re-arranging
the input-output positions «modulo naturality». Now homotopy\slash
knots\slash proof-nets may be held as (faithfull or almost-faithfull)
semantical techniques \\ («descent») to do this «categorial logic»,
and the homotopy itself may be programmed in specialized grammars
(for example \texttt{HOTT} or else). \ \ \ \ 2. The common assumption
that \texttt{catC( $-$ , X )} is dual to \texttt{catC( Y , $-$ )}
is FALSIFIED. This falsification originates from the description of
the composition as some binary form instead of as some functional
form which is programmed\slash encoded\slash enriched onto the computer.
Then get some new thing which is named «polymorphism» from which to
define «polymorph category». This is the only-ever real description
and deduction of the yoneda lemma, which says that the image of \texttt{polyF}
(which is injective and contained in natural transformations) also
contains all natural transformations. \ \ \ \ 3. Some polymorph category
(functor) is given by \texttt{polyF}, which is commonl\texttt{y (
(\_1 \_3) o>F \_2 )} , polymorph in \texttt{V} and polymorph in \texttt{A}
:

\texttt{Variable obF : Type. Variable polyF00 : obF $\to$ obF $\to$
obV.}

\texttt{Notation \textquotedbl{}F{[}0 B $\leadsto$ A {]}0\textquotedbl{}
:= (polyF00 B A) (at level 25).}

\texttt{Parameter polyF : forall (B : obF), forall (V : oybV) (A : obF),}

\texttt{V(0 V $\vdash$ F{[}0 B $\leadsto$ A {]}0 )0 }$\to$

\texttt{forall X : obF, V(0 F{[}0 A $\leadsto$ X {]}0 $\vdash$ {[}0
V $\leadsto$ F{[}0 B $\leadsto$ X {]}0 {]}0 )0.}

4. And to get polymorph functor, \texttt{catV{[} V , catB{[} B , F
$-$ {]} {]} : catA $\to$ catV} . And to get polymorph transformation
a-la-dosen, \texttt{phi \_f : catV( V , catB{[} B , G A {]} ) $\to$
catV( V , catB{[} B , H A {]} )} . \ \ \ \ 5. And finally one shall
relate the earlier «naturality of this particular metatransformation
inside \texttt{catV}» to this new general «polymorphism» for the corresponding
polytransformation. This requires to define composition of polyfunctors,
where one discovers that any functor is in reality alone \texttt{polyF}
for some sequence of any length of mappings on objects, \texttt{catA
$\to$ catB $\to$ catC $\to$ ...} , where \texttt{catA} is encoded
(enriched) in \texttt{catB}, and \texttt{catB} is encoded in \texttt{catC},
... and \texttt{catC} is encoded in \texttt{catV} which finally exits
into sets/types/classifiers; this may solve the \texttt{n}-category
yoneda question...\ \ \ \ 6. These earlier texts referring to Maclane
associativity coherence and Dosen semiassociativity coherence and
Dosen cut elimination for adjunctions and Chlipala ur/web database
programming are all related to this present text of the «\texttt{1337777.OOO
Solution Programme}».
\end{document}
